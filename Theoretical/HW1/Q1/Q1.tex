% Defining document class and essential packages
\documentclass[a4paper,12pt]{article}
\usepackage{amsmath}
\usepackage{amssymb}
\usepackage{geometry}
\geometry{margin=1in}
\usepackage{graphicx}
\usepackage{parskip}

% Setting up the title
\title{Statistical Pattern Recognition - Homework 1 \\ Question 1}
\author{Vahid Maleki \\ Student ID: 40313004}
\date{October 11, 2025}

% Beginning the document
\begin{document}
	\maketitle
	
	% Introducing the problem statement
	\section*{Question 1}
	Given the prior probabilities \( p(w_1) = p(w_2) \) and the class-conditional densities functions \( p(x|w_1) = N_x(0, 1) \) and \( p(x|w_2) = N_x(1, 2) \), calculate and draw the posterior probabilities \( p(w_1|x) \) and \( p(w_2|x) \).
	
	% Providing the solution step-by-step
	\section*{Solution}
	
	% Step 1: Stating the given information
	Since \( p(w_1) = p(w_2) \), and the priors must sum to 1, we have:
	\[
	p(w_1) = p(w_2) = 0.5.
	\]
	The class-conditional densities functions are:
	\[
	p(x|w_1) = N_x(0, 1) = \frac{1}{\sqrt{2\pi}} \exp\left(-\frac{x^2}{2}\right),
	\]
	\[
	p(x|w_2) = N_x(1, 2) = \frac{1}{\sqrt{2\pi \cdot 2}} \exp\left(-\frac{(x-1)^2}{2 \cdot 2}\right) = \frac{1}{\sqrt{4\pi}} \exp\left(-\frac{(x-1)^2}{4}\right).
	\]
	
	% Step 2: Applying Bayes' theorem
	To find the posterior probabilities \( p(w_1|x) \) and \( p(w_2|x) \), we use Bayes' theorem:
	\[
	p(w_i|x) = \frac{p(x|w_i) p(w_i)}{p(x)}, \quad i = 1, 2,
	\]
	where \( p(x) \) is the evidence, given by:
	\[
	p(x) = p(x|w_1) p(w_1) + p(x|w_2) p(w_2).
	\]
	
	Substitute the known values:
	\[
	p(x) = p(x|w_1) \cdot 0.5 + p(x|w_2) \cdot 0.5 = 0.5 p(x|w_1) + 0.5 p(x|w_2).
	\]
	
	So:
	\[
	p(w_1|x) = \frac{p(x|w_1) \cdot 0.5}{p(x)}, \quad p(w_2|x) = \frac{p(x|w_2) \cdot 0.5}{p(x)}.
	\]
	
	% Step 3: Simplifying the posterior probabilities
	Since the factor of 0.5 appears in both numerator and denominator, we can simplify:
	\[
	p(w_1|x) = \frac{p(x|w_1)}{p(x|w_1) + p(x|w_2)}, \quad p(w_2|x) = \frac{p(x|w_2)}{p(x|w_1) + p(x|w_2)}.
	\]
	
	Substitute the class-conditional densities functions:
	\[
	p(x|w_1) = \frac{1}{\sqrt{2\pi}} \exp\left(-\frac{x^2}{2}\right),
	\]
	\[
	p(x|w_2) = \frac{1}{\sqrt{4\pi}} \exp\left(-\frac{(x-1)^2}{4}\right).
	\]
	
	Thus:
	\[
	p(w_1|x) = \frac{\frac{1}{\sqrt{2\pi}} \exp\left(-\frac{x^2}{2}\right)}{\frac{1}{\sqrt{2\pi}} \exp\left(-\frac{x^2}{2}\right) + \frac{1}{\sqrt{4\pi}} \exp\left(-\frac{(x-1)^2}{4}\right)}.
	\]
	
	Simplify the denominator:
	\[
	\sqrt{4\pi} = \sqrt{2 \cdot 2\pi} = \sqrt{2} \cdot \sqrt{2\pi},
	\]
	so:
	\[
	\frac{1}{\sqrt{4\pi}} = \frac{1}{\sqrt{2} \cdot \sqrt{2\pi}}.
	\]
	
	To simplify \( p(w_1|x) \), multiply numerator and denominator by \( \sqrt{2\pi} \):
	\[
	p(w_1|x) = \frac{\exp\left(-\frac{x^2}{2}\right)}{\exp\left(-\frac{x^2}{2}\right) + \frac{1}{\sqrt{2}} \exp\left(-\frac{(x-1)^2}{4}\right)}.
	\]
	
	Similarly:
	\[
	p(w_2|x) = \frac{\frac{1}{\sqrt{2}} \exp\left(-\frac{(x-1)^2}{4}\right)}{\exp\left(-\frac{x^2}{2}\right) + \frac{1}{\sqrt{2}} \exp\left(-\frac{(x-1)^2}{4}\right)}.
	\]
	
	% Step 4: Alternative form using logistic function
	Notice that the posterior probabilities can be expressed using the logistic function. Let:
	\[
	a = \frac{p(x|w_1)}{p(x|w_2)} = \frac{\frac{1}{\sqrt{2\pi}} \exp\left(-\frac{x^2}{2}\right)}{\frac{1}{\sqrt{4\pi}} \exp\left(-\frac{(x-1)^2}{4}\right)} = \sqrt{2} \exp\left(-\frac{x^2}{2} + \frac{(x-1)^2}{4}\right).
	\]
	
	Simplify the exponent:
	\[
	-\frac{x^2}{2} + \frac{(x-1)^2}{4} = -\frac{x^2}{2} + \frac{x^2 - 2x + 1}{4} = -\frac{2x^2}{4} + \frac{x^2 - 2x + 1}{4} = \frac{-x^2 - 2x + 1}{4}.
	\]
	
	Thus:
	\[
	a = \sqrt{2} \exp\left(\frac{-x^2 - 2x + 1}{4}\right).
	\]
	
	Then:
	\[
	p(w_1|x) = \frac{a}{a + 1}, \quad p(w_2|x) = \frac{1}{a + 1}.
	\]
	
	This form is suitable for numerical computation.
	
	% Step 5: Plotting the posterior probabilities
	\section*{Plotting \( p(w_1|x) \) and \( p(w_2|x) \)}
	To draw \( p(w_1|x) \) and \( p(w_2|x) \), plot the functions:
	\[
	p(w_1|x) = \frac{\exp\left(-\frac{x^2}{2}\right)}{\exp\left(-\frac{x^2}{2}\right) + \frac{1}{\sqrt{2}} \exp\left(-\frac{(x-1)^2}{4}\right)},
	\]
	\[
	p(w_2|x) = \frac{\frac{1}{\sqrt{2}} \exp\left(-\frac{(x-1)^2}{4}\right)}{\exp\left(-\frac{x^2}{2}\right) + \frac{1}{\sqrt{2}} \exp\left(-\frac{(x-1)^2}{4}\right)}.
	\]
	
	These can be plotted using computational tools like Python (with matplotlib) or MATLAB over a range of \( x \), e.g., \( x \in [-5, 5] \). The plot will show:
	- \( p(w_1|x) \) starting high for negative \( x \), decreasing as \( x \) increases.
	- \( p(w_2|x) \) starting low for negative \( x \), increasing as \( x \) increases.
	- The curves intersect where \( p(w_1|x) = p(w_2|x) = 0.5 \), which occurs when \( p(x|w_1) = p(x|w_2) \).
	
	To find the intersection, solve:
	\[
	\frac{1}{\sqrt{2\pi}} \exp\left(-\frac{x^2}{2}\right) = \frac{1}{\sqrt{4\pi}} \exp\left(-\frac{(x-1)^2}{4}\right).
	\]
	
	This requires numerical methods, but the plot will visually confirm the behavior.
	
	% Concluding the solution
	The final expressions for the posterior probabilities are provided above, and the plots can be generated using the given formulas in a suitable programming environment.
	
\end{document}