\documentclass[12pt,a4paper]{article}
\usepackage[utf8]{inputenc}
\usepackage{xepersian}
\settextfont{Amiri}
\setlatintextfont{Times New Roman}
\usepackage{amsmath}
\usepackage{amsfonts}
\usepackage{amssymb}
\usepackage{geometry}
\geometry{margin=2.5cm}

\title{تکلیف چهارم درس شناسایی الگو}
\author{وحید ملکی \\ شماره دانشجویی: 40313004}
\date{7 نوامبر 2025}

\begin{document}
	
	\maketitle
	
	\section{سوال 15}
	
	یک مسئله دسته‌بندی دوکلاسه در فضای یک‌بعدی داریم. توابع چگالی احتمال برای هر کلاس این‌طوریه:
	
	\[
	p(x|c_1) = 
	\begin{cases}
		0 & x < 0 \\
		\theta_1 e^{-\theta_1 x} & x \ge 0
	\end{cases}
	\]
	
	\[
	p(x|c_2) = 
	\begin{cases}
		0 & x < 0 \\
		\theta_2 e^{-\theta_2 x} & x \ge 0
	\end{cases}
	\]
	
	داده‌های آموزشی: کلاس ۱: $D_1 = \{3, 5\}$، کلاس ۲: $D_2 = \{6, 9, 12\}$.
	
	\subsection{الف) برآورد $\theta_1$ و $\theta_2$ با MLE}
	
	برای توزیع نمایی $p(x|\theta) = \theta e^{-\theta x}$، تابع لگاریتم درست‌نمایی برای $N$ نمونه $\mathcal{L}(\theta) = N \ln \theta - \theta \sum x_i$ هست.
	
	مشتق: $\frac{\partial \mathcal{L}}{\partial \theta} = \frac{N}{\theta} - \sum x_i = 0 \implies \hat{\theta} = \frac{N}{\sum x_i} = \frac{1}{\bar{x}}$.
	
	برای کلاس ۱: $N_1=2$، $\bar{x}_1 = \frac{3+5}{2} = 4$، پس $\hat{\theta}_1 = \frac{1}{4} = 0.25$.
	
	برای کلاس ۲: $N_2=3$، $\bar{x}_2 = \frac{6+9+12}{3} = 9$، پس $\hat{\theta}_2 = \frac{1}{9} \approx 0.111$.
	
	\subsection{ب) نواحی و مرز تصمیم با دسته‌بند ML}
	
	فرض می‌کنیم احتمال پیشین کلاس‌ها برابر باشه ($P(c_1)=P(c_2)$). تصمیم بر اساس $p(x|c_1) > p(x|c_2)$ برای کلاس ۱.
	
	مرز تصمیم جایی که $p(x|c_1) = p(x|c_2)$، یعنی $\hat{\theta}_1 e^{-\hat{\theta}_1 x} = \hat{\theta}_2 e^{-\hat{\theta}_2 x}$.
	
	جایگذاری: $\frac{1}{4} e^{-x/4} = \frac{1}{9} e^{-x/9}$.
	
	$$\frac{e^{-x/4}}{e^{-x/9}} = \frac{1/9}{1/4} = \frac{4}{9}$$
	
	$$e^{x(1/9 - 1/4)} = \frac{4}{9}$$
	
	$$e^{-x (5/36)} = \frac{4}{9}$$
	
	$\ln$ هر دو طرف: $$-x \frac{5}{36} = \ln(\frac{4}{9}) = \ln 4 - \ln 9$$
	
	$$x \frac{5}{36} = \ln 9 - \ln 4$$
	
	$$x = \frac{36}{5} (\ln 9 - \ln 4) \approx 7.2 \times (2.197 - 1.386) \approx 7.2 \times 0.811 \approx 5.84$$
	
	برای چک کردن ناحیه: در $x=0$، $p(0|c_1)=0.25 > p(0|c_2) \approx 0.111$، پس ناحیه $0 \le x < 5.84$ برای $c_1$ و $x > 5.84$ برای $c_2$.
	
	(برای $x<0$، هر دو صفره و تصمیم تعریف نشده.)
	
\end{document}
