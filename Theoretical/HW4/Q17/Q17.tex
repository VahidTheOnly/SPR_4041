\documentclass[12pt,a4paper]{article}
\usepackage[utf8]{inputenc}
\usepackage{xepersian}
\settextfont{Amiri}
\setlatintextfont{Times New Roman}
\usepackage{amsmath}
\usepackage{amsfonts}
\usepackage{amssymb}
\usepackage{geometry}
\geometry{margin=2.5cm}
\title{تکلیف چهارم درس شناسایی الگو}
\author{وحید ملکی \\ شماره دانشجویی: 40313004}
\date{7 نوامبر 2025}
\begin{document}
	\maketitle
	
	% ----------------------------------------------------------------
	\section{سوال 17}
	% ----------------------------------------------------------------
	
	متغیر تصادفی $ x $ از توزیعی با تابع چگالی احتمال زیر پیروی می‌کند:
	$$ 
	p(x; \theta) = \frac{\theta^3}{2} x^2 e^{-\theta x},\quad x>0,\ \theta>0
	$$
	فرض کنید $ N $ نمونه‌ی مستقل $ x_1,x_2,\dots,x_N $ از این توزیع در اختیار است.  
	مطلوب است تخمین بیشینه‌ی درست‌نمایی (Maximum Likelihood Estimation) برای پارامتر $ \theta $ محاسبه شود.
	
	\subsection{جواب}
	
	\subsubsection{۱. تشکیل تابع درست‌نمایی (Likelihood Function)}
	
	تابع درست‌نمایی $ L(\theta) $ برابر با حاصل‌ضرب تابع چگالی احتمال برای هر یک از $ N $ نمونه‌ی مستقل است:
	$$ 
	L(\theta)=P(x_1,\dots,x_N|\theta)=\prod_{i=1}^N p(x_i;\theta)
	$$
	با جایگذاری $ p(x;\theta) $ داده‌شده:
	$$ 
	L(\theta)=\prod_{i=1}^N \left(\frac{\theta^3}{2}\,x_i^2\,e^{-\theta x_i}\right)
	$$
	ساده‌سازی عبارت:
	$$ 
	L(\theta)=\left(\frac{\theta^3}{2}\right)^N
	\Bigl(\prod_{i=1}^N x_i^2\Bigr)
	\Bigl(\prod_{i=1}^N e^{-\theta x_i}\Bigr)
	=\frac{\theta^{3N}}{2^N}\,
	\Bigl(\prod_{i=1}^N x_i^2\Bigr)\,
	e^{-\theta\sum_{i=1}^N x_i}
	$$
	
	\subsubsection{۲. تشکیل تابع لگاریتم درست‌نمایی (Log-Likelihood)}
	
	$$ 
	\mathcal{L}(\theta)=\ln L(\theta)
	=\ln\!\left[\frac{\theta^{3N}}{2^N}
	\Bigl(\prod_{i=1}^N x_i^2\Bigr)
	e^{-\theta\sum_{i=1}^N x_i}\right]
	$$
	با استفاده از خواص لگاریتم:
	$$ 
	\mathcal{L}(\theta)
	=\ln\!\left(\frac{\theta^{3N}}{2^N}\right)
	+\ln\!\left(\prod_{i=1}^N x_i^2\right)
	+\ln\!\left(e^{-\theta\sum_{i=1}^N x_i}\right)
	$$
	$$ 
	=3N\ln\theta-N\ln 2
	+\sum_{i=1}^N\ln(x_i^2)
	-\theta\sum_{i=1}^N x_i
	=3N\ln\theta-N\ln 2
	+2\sum_{i=1}^N\ln x_i
	-\theta\sum_{i=1}^N x_i
	$$
	
	\subsubsection{۳. مشتق‌گیری و یافتن نقطه اکسترمم}
	
	مشتق نسبت به $ \theta $:
	$$ 
	\frac{d\mathcal{L}}{d\theta}
	=\frac{d}{d\theta}\!\left[
	3N\ln\theta-N\ln 2+2\sum_{i=1}^N\ln x_i
	-\theta\sum_{i=1}^N x_i
	\right]
	=\frac{3N}{\theta}-\sum_{i=1}^N x_i
	$$
	قرار دادن مشتق برابر صفر:
	$$ 
	\frac{3N}{\theta}-\sum_{i=1}^N x_i=0
	\qquad\Longrightarrow\qquad
	\frac{3N}{\theta}=\sum_{i=1}^N x_i
	$$
	
	\subsubsection{۴. محاسبه تخمین $ \hat{\theta}_{MLE} $}
	
	$$ 
	\theta=\frac{3N}{\sum_{i=1}^N x_i}
	\qquad\Longrightarrow\qquad
	\hat{\theta}_{MLE}=\frac{3N}{\sum_{i=1}^N x_i}
	$$
	اگر میانگین نمونه‌ها را $ \bar{x}=\frac{1}{N}\sum_{i=1}^N x_i $ تعریف کنیم، می‌توان نوشت:
	$$ 
	\hat{\theta}_{MLE}=\frac{3}{\bar{x}}
	$$
	
\end{document}