\documentclass{article}
\usepackage{amsmath}
\usepackage{amssymb}
\usepackage{geometry}
\geometry{a4paper, margin=1in}

\title{Q18 - HW2: Pattern Recognition}
\author{Vahid Maleki \\ Student ID: 40313004}
\date{October 18, 2025}

\begin{document}
	
	\maketitle
	
	\section{Question 18}
	
	Given the following \textbf{covariance matrix}:
	
	\[
	\Sigma =
	\begin{bmatrix}
		1 & 0 & 0 & 0 & 0 & 0 \\
		0 & 4 & 0 & 0 & 0 & 3 \\
		0 & 0 & 1 & 0 & 0 & 0 \\
		0 & 0 & 0 & 2 & 0 & 0 \\
		0 & 0 & 0 & 0 & 3 & 0 \\
		0 & 3 & 0 & 0 & 0 & 5
	\end{bmatrix}
	\]
	
	To ensure that \textbf{less than 30\% of the total information (variance)} is lost, what is the \textbf{minimum number of eigenvectors} that must be retained in the \textbf{PCA (Principal Component Analysis)} algorithm?
	
	\subsection{Solution}
	
	To determine the minimum number of eigenvectors to retain in PCA while ensuring that less than 30\% of the total variance is lost, we need to compute the eigenvalues of the covariance matrix $\Sigma$, calculate the total variance, and find the smallest number of eigenvalues (in descending order) that account for at least 70\% of the total variance.
	
	\textbf{Step 1: Find the Eigenvalues}
	
	The covariance matrix $\Sigma$ is block-diagonal with a $2 \times 2$ block in rows/columns 2 and 6, and the other diagonal entries are eigenvalues directly. The block is:
	
	\[
	\begin{bmatrix}
		4 & 3 \\
		3 & 5
	\end{bmatrix}
	\]
	
	Compute the eigenvalues of this block by solving:
	
	\[
	\det\left(\begin{bmatrix} 4 - \lambda & 3 \\ 3 & 5 - \lambda \end{bmatrix}\right) = 0
	\]
	
	\[
	(4 - \lambda)(5 - \lambda) - 9 = \lambda^2 - 9\lambda + 20 - 9 = \lambda^2 - 9\lambda + 11 = 0
	\]
	
	\[
	\lambda = \frac{9 \pm \sqrt{81 - 44}}{2} = \frac{9 \pm \sqrt{37}}{2}
	\]
	
	Since $\sqrt{37} \approx 6.08$, the eigenvalues are:
	
	\[
	\lambda_1 \approx \frac{9 + 6.08}{2} \approx 7.54, \quad \lambda_2 \approx \frac{9 - 6.08}{2} \approx 1.46
	\]
	
	The other diagonal entries correspond to eigenvalues for the remaining dimensions:
	
	- Row 1: $1$
	- Row 3: $1$
	- Row 4: $2$
	- Row 5: $3$
	
	Thus, the full set of eigenvalues is:
	
	\[
	1, 1, 2, 3, 7.54, 1.46
	\]
	
	\textbf{Step 2: Sort Eigenvalues}
	
	Sort the eigenvalues in descending order:
	
	\[
	7.54, 3, 2, 1.46, 1, 1
	\]
	
	\textbf{Step 3: Compute Total Variance}
	
	The total variance is the sum of all eigenvalues:
	
	\[
	7.54 + 3 + 2 + 1.46 + 1 + 1 = 16
	\]
	
	\textbf{Step 4: Determine Minimum Eigenvectors for 70\% Variance}
	
	To ensure that less than 30\% of the variance is lost, we need to retain at least 70\% of the total variance:
	
	\[
	70\% \text{ of } 16 = 0.7 \times 16 = 11.2
	\]
	
	Add the eigenvalues in descending order until the cumulative sum is at least 11.2:
	
	- First eigenvalue: $7.54$
	- Second eigenvalue: $7.54 + 3 = 10.54$
	- Third eigenvalue: $10.54 + 2 = 12.54$
	
	After three eigenvalues, the cumulative variance is 12.54, which exceeds 11.2.
	
	\textbf{Step 5: Conclusion}
	
	To ensure that at least 70\% of the total variance is retained (i.e., less than 30\% is lost), we need to keep the top three eigenvectors.
	
	\[
	\boxed{3}
	\]
	
\end{document}