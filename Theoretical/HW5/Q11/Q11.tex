\documentclass[12pt,a4paper]{article}
\usepackage[utf8]{inputenc}
\usepackage{xepersian}
\settextfont{Amiri}
\setlatintextfont{Times New Roman}
\usepackage{amsmath}
\usepackage{amsfonts}
\usepackage{amssymb}
\usepackage{geometry}
\usepackage{enumitem}
\geometry{margin=2.5cm}

\title{تکلیف پنجم درس شناسایی الگو}
\author{وحید ملکی \\ شماره دانشجویی: 40313004}
\date{\today}

\begin{document}
	
	\maketitle
	
	\section*{سؤال ۱۱}
	
	فرض کنید نمونه‌های زیر مشاهده شده‌اند:  
	\textbf{کلاس اول (\(\omega_1\))}: \(-1, -2, 3, 3, 7, 7\) \quad (\(n_1 = 6\))  
	\textbf{کلاس دوم (\(\omega_2\))}: \(-3, -2, 3, 5, 8\) \quad (\(n_2 = 5\))
	
	از \textbf{تخمین‌گر پارزن} با \textbf{کرنل صفر-یک} و پهنای پنجره \(h = 2\) استفاده شده است.  
	هدف: نشان دادن اینکه نمونه‌ی جدید در نقطه \(x = 4\) با استفاده از \textbf{کلاسیفایر بیشینهٔ شباهت (Maximum Likelihood)} در \textbf{کلاس اول} قرار می‌گیرد.
	
	\subsection*{تعریف کرنل}
	کرنل داده‌شده:
	\[
	\varphi(x) = 
	\begin{cases}
		\dfrac{a}{2} & |x| < a \\
		0 & \text{در غیر این صورت}
	\end{cases}
	\]
	با \(a = 2\):
	\[
	\varphi(x) = 
	\begin{cases}
		1 & |x| < 2 \\
		0 & \text{در غیر این صورت}
	\end{cases}
	\]
	
	\subsection*{فرمول تخمین‌گر پارزن}
	\[
	p(x \mid \omega_j) = \frac{1}{n_j h} \sum_{i=1}^{n_j} \varphi\left( \frac{x - x_i}{h} \right)
	\]
	
	\subsection*{محاسبه \(p(x=4 \mid \omega_1)\)}
	\[
	p(4 \mid \omega_1) = \frac{1}{6 \times 2} \sum_{i=1}^{6} \varphi\left( \frac{4 - x_i}{2} \right) = \frac{1}{12} \sum_{i=1}^{6} \varphi\left( \frac{4 - x_i}{2} \right)
	\]
	
	شرط فعال شدن کرنل:
	\[
	\left| \frac{4 - x_i}{2} \right| < 2 \quad \Rightarrow \quad |4 - x_i| < 4 \quad \Rightarrow \quad 0 < x_i < 8
	\]
	
	نمونه‌های کلاس اول در بازه \((0, 8)\):  
	\(x_i = 3, 3, 7, 7\) → **۴ نمونه**
	
	\[
	\sum = 4 \quad \Rightarrow \quad p(4 \mid \omega_1) = \frac{1}{12} \times 4 = \frac{4}{12} = \boxed{\dfrac{1}{3}}
	\]
	
	\subsection*{محاسبه \(p(x=4 \mid \omega_2)\)}
	\[
	p(4 \mid \omega_2) = \frac{1}{5 \times 2} \sum_{i=1}^{5} \varphi\left( \frac{4 - x_i}{2} \right) = \frac{1}{10} \sum_{i=1}^{5} \varphi\left( \frac{4 - x_i}{2} \right)
	\]
	
	شرط همان است: \(0 < x_i < 8\)  
	نمونه‌های کلاس دوم در این بازه:  
	\(x_i = 3, 5\) → **۲ نمونه**
	
	\[
	\sum = 2 \quad \Rightarrow \quad p(4 \mid \omega_2) = \frac{1}{10} \times 2 = \frac{2}{10} = \boxed{\dfrac{1}{5}}
	\]
	
	\subsection*{تصمیم‌گیری با کلاسیفایر بیشینهٔ شباهت}
	\[
	\hat{\omega} = \arg\max_{\omega_j} p(x \mid \omega_j)
	\]
	
	مقایسه:
	\[
	p(4 \mid \omega_1) = \frac{1}{3} \approx 0.333, \quad p(4 \mid \omega_2) = \frac{1}{5} = 0.2
	\]
	
	\[
	\boxed{\dfrac{1}{3} > \dfrac{1}{5}}
	\]
	
	بنابراین، نمونه‌ی جدید در نقطه \(x = 4\) به \textbf{کلاس اول} اختصاص می‌یابد.
	
	\begin{center}
		\begin{tabular}{|c|c|c|}
			\hline
			\textbf{کلاس} & \textbf{تعداد نمونه در پنجره} & \(p(4 \mid \omega_j)\) \\
			\hline
			\(\omega_1\) & ۴ & \(\dfrac{1}{3}\) \\
			\(\omega_2\) & ۲ & \(\dfrac{1}{5}\) \\
			\hline
			\textbf{تصمیم نهایی} & \multicolumn{2}{c|}{\textbf{کلاس ۱}} \\
			\hline
		\end{tabular}
	\end{center}
	
\end{document}