\documentclass[12pt,a4paper]{article}
\usepackage[utf8]{inputenc}
\usepackage{amsmath}
\usepackage{amssymb}
\usepackage{booktabs}
\usepackage{geometry}
\geometry{margin=1in}

\begin{document}
	
	\begin{center}
		\textbf{\large Pattern Recognition -- Homework 3}\\
		\vspace{0.2cm}
		Vahid Maleki \quad Student ID: 40313004
	\end{center}
	
	\vspace{1cm}
	
	\section*{Question 8}
	
	\subsection*{Problem Statement}
	
	We have a two-class problem with feature $x \in \{1,2\}$. The class-conditional probabilities are:
	
	\begin{center}
		\begin{tabular}{c|c|c}
			\toprule
			$x$ & $p(x|\omega_1)$ & $p(x|\omega_2)$ \\
			\midrule
			1   & $1/3$           & $2/3$           \\
			2   & $2/3$           & $1/3$           \\
			\bottomrule
		\end{tabular}
	\end{center}
	
	The loss matrix is
	\[
	L = \begin{pmatrix}
		0 & 2 \\
		1 & 0
	\end{pmatrix},
	\quad
	\lambda_{ki} = \text{cost of deciding } \omega_i \text{ when true class is } \omega_k.
	\]
	
	The conditional risk for class $\omega_k$ is
	\[
	r_k = \sum_{i=1}^2 \lambda_{ki} \sum_{x \in R_i} p(x|\omega_k).
	\]
	
	\subsection*{a) Conditional Risks for All Possible Partitions}
	
	Since $x$ is discrete, we consider the four possible decision rules.
	
	\paragraph{Rule 1: $R_1 = \{1,2\}$, $R_2 = \emptyset$ (always choose $\omega_1$)}
	
	For $\omega_1$:
	\[
	r_1 = 0 \cdot (1/3 + 2/3) + 2 \cdot 0 = 0.
	\]
	
	For $\omega_2$:
	\[
	r_2 = 1 \cdot (2/3 + 1/3) + 0 \cdot 0 = 1.
	\]
	
	\paragraph{Rule 2: $R_1 = \emptyset$, $R_2 = \{1,2\}$ (always choose $\omega_2$)}
	
	For $\omega_1$:
	\[
	r_1 = 0 \cdot 0 + 2 \cdot (1/3 + 2/3) = 2.
	\]
	
	For $\omega_2$:
	\[
	r_2 = 1 \cdot 0 + 0 \cdot (2/3 + 1/3) = 0.
	\]
	
	\paragraph{Rule 3: $R_1 = \{1\}$, $R_2 = \{2\}$}
	
	For $\omega_1$:
	\[
	r_1 = 0 \cdot (1/3) + 2 \cdot (2/3) = \frac{4}{3}.
	\]
	
	For $\omega_2$:
	\[
	r_2 = 1 \cdot (2/3) + 0 \cdot (1/3) = \frac{2}{3}.
	\]
	
	\paragraph{Rule 4: $R_1 = \{2\}$, $R_2 = \{1\}$}
	
	For $\omega_1$:
	\[
	r_1 = 0 \cdot (2/3) + 2 \cdot (1/3) = \frac{2}{3}.
	\]
	
	For $\omega_2$:
	\[
	r_2 = 1 \cdot (1/3) + 0 \cdot (2/3) = \frac{1}{3}.
	\]
	
	\subsection*{b) Average Risk with $P(\omega_1)=1/4$, $P(\omega_2)=3/4$}
	
	The average risk is $r = r_1 P(\omega_1) + r_2 P(\omega_2)$.
	
	\begin{itemize}
		\item Rule 1: $r = 0 \cdot \frac{1}{4} + 1 \cdot \frac{3}{4} = \frac{3}{4}$.
		\item Rule 2: $r = 2 \cdot \frac{1}{4} + 0 \cdot \frac{3}{4} = \frac{1}{2}$.
		\item Rule 3: $r = \frac{4}{3} \cdot \frac{1}{4} + \frac{2}{3} \cdot \frac{3}{4} = \frac{4}{12} + \frac{6}{12} = \frac{5}{6}$.
		\item Rule 4: $r = \frac{2}{3} \cdot \frac{1}{4} + \frac{1}{3} \cdot \frac{3}{4} = \frac{2}{12} + \frac{3}{12} = \frac{5}{12}$.
	\end{itemize}
	
	\subsection*{c) Optimal Bayes Rule}
	
	The risks are $\frac{3}{4}$, $\frac{1}{2}$, $\frac{5}{6}$, $\frac{5}{12}$. The smallest is $\frac{5}{12}$, achieved by Rule 4.
	
	Thus the optimal decision is:
	\begin{itemize}
		\item If $x=1$, decide $\omega_2$.
		\item If $x=2$, decide $\omega_1$.
	\end{itemize}
	
\end{document}