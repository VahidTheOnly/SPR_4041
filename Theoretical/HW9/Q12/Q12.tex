\documentclass[12pt,a4paper]{article}
\usepackage[utf8]{inputenc}
\usepackage{xepersian}
\settextfont{Amiri}
\setlatintextfont{Times New Roman}
\usepackage{amsmath}
\usepackage{amsfonts}
\usepackage{amssymb}
\usepackage{geometry}
\usepackage{graphicx}
\usepackage{float}
\geometry{margin=2.5cm}

\title{تکلیف نهم درس شناسایی الگو}
\author{وحید ملکی \\ شماره دانشجویی: 40313004}
\date{\today}

\begin{document}
	\maketitle
	
	\section*{سوال ۱۲}
	در این سوال یک معماری شبکه عصبی کانولوشنی (CNN) ساده ارائه شده است.
	روابط حاکم بر شبکه به شرح زیر است:
	\begin{itemize}
		\item \textbf{لایه اول (کانولوشن):} شامل یک فیلتر با سایز ۳ است. وزن‌های فیلتر با $v_1, v_2, v_3$ نمایش داده می‌شوند.
		$$ h_i = s(u_i) = s\left(\sum_{j=1}^{3} v_j x_{i+j-1}\right) $$
		که در آن $s(x)$ تابع فعالیت سیگموئید است و $u_i$ ورودی خالص به نورون $h_i$ می‌باشد.
		\item \textbf{لایه دوم (تمام متصل):} خروجی $z$ ترکیب خطی نورون‌های مخفی است.
		$$ z = \sum_{i=1}^{4} w_i h_i $$
		\item \textbf{تابع هزینه:} مربع خطا تعریف شده است.
		$$ R = (y - z)^2 $$
	\end{itemize}
	
	\subsection*{الف) تعداد کل پارامترهای شبکه}
	در شبکه‌های کانولوشنی، اصل «اشتراک وزن» (\lr{Weight Sharing}) برقرار است. این یعنی وزن‌های استفاده شده برای محاسبه هر $h_i$ یکسان هستند.
	\begin{itemize}
		\item \textbf{لایه اول (کانولوشن):} اگرچه خطوط زیادی در شکل رسم شده است، اما تنها یک فیلتر با سایز ۳ وجود دارد. بنابراین تنها ۳ پارامتر $v_1, v_2, v_3$ در این لایه وجود دارد.
		\item \textbf{لایه دوم (تمام متصل):} هر نورون مخفی ($h_1$ تا $h_4$) با یک وزن منحصر به فرد به $z$ متصل است. پس ۴ پارامتر $w_1, w_2, w_3, w_4$ در اینجا داریم.
		\item \textbf{بایاس:} طبق صورت سوال، ترم‌های بایاس وجود ندارند.
	\end{itemize}
	\textbf{تعداد کل پارامترها:}
	$$ 3 (\text{برای } v) + 4 (\text{برای } w) = 7 \text{ پارامتر} $$
	
	\subsection*{ب) محاسبه مشتق تابع هزینه نسبت به $w_i$}
	برای محاسبه $\frac{\partial R}{\partial w_i}$ از قاعده زنجیره‌ای استفاده می‌کنیم:
	$$ \frac{\partial R}{\partial w_i} = \frac{\partial R}{\partial z} \cdot \frac{\partial z}{\partial w_i} $$
	
	۱. مشتق هزینه نسبت به خروجی شبکه ($z$):
	$$ \frac{\partial R}{\partial z} = \frac{\partial}{\partial z}(y - z)^2 = 2(y - z)(-1) = -2(y - z) $$
	
	۲. مشتق خروجی نسبت به وزن $w_i$:
	با توجه به رابطه $z = \sum_{k=1}^{4} w_k h_k$:
	$$ \frac{\partial z}{\partial w_i} = h_i $$
	
	۳. ترکیب روابط:
	$$ \frac{\partial R}{\partial w_i} = -2(y - z) h_i $$
	
	\subsection*{ج) فرم برداری رابطه قبل}
	می‌خواهیم گرادیان $\frac{\partial R}{\partial \mathbf{w}}$ را محاسبه کنیم که در آن $\mathbf{w} = [w_1, w_2, w_3, w_4]^T$ است.
	
	اگر بردار خروجی‌های لایه مخفی را $\mathbf{h} = [h_1, h_2, h_3, h_4]^T$ در نظر بگیریم، با توجه به اینکه عبارت $-2(y-z)$ یک اسکالر است که در تمام مولفه‌ها ضرب می‌شود:
	$$ \frac{\partial R}{\partial \mathbf{w}} = -2(y - z) \mathbf{h} $$
	
	\subsection*{د) محاسبه مشتق تابع هزینه نسبت به $v_j$}
	این قسمت نیازمند دقت بیشتری است زیرا پارامتر $v_j$ در محاسبه تمام $h_i$ها مشترک است (\lr{Weight Sharing}). بنابراین تغییر در $v_j$ روی تمامی $h_1$ تا $h_4$ و در نهایت روی $z$ اثر می‌گذارد. طبق قاعده زنجیره‌ای چند متغیره باید روی تمام مسیرهای ممکن (تمام $i$ها) جمع ببندیم:
	
	$$ \frac{\partial R}{\partial v_j} = \frac{\partial R}{\partial z} \sum_{i=1}^{4} \left( \frac{\partial z}{\partial h_i} \cdot \frac{\partial h_i}{\partial v_j} \right) $$
	
	اجزای رابطه را محاسبه می‌کنیم:
	\begin{enumerate}
		\item $\frac{\partial R}{\partial z} = -2(y - z)$ (مشابه قسمت ب)
		\item $\frac{\partial z}{\partial h_i} = w_i$
		\item محاسبه $\frac{\partial h_i}{\partial v_j}$:
		می‌دانیم $h_i = s(u_i)$ که در آن $u_i = \sum_{k=1}^{3} v_k x_{i+k-1}$.
		$$ \frac{\partial h_i}{\partial v_j} = \frac{\partial s(u_i)}{\partial u_i} \cdot \frac{\partial u_i}{\partial v_j} $$
		$$ \frac{\partial h_i}{\partial v_j} = s'(u_i) \cdot x_{i+j-1} $$
		(توضیح: ضریب $v_j$ در عبارت $u_i$ برابر است با $x_{i+j-1}$).
	\end{enumerate}
	
	جایگذاری در رابطه کلی:
	$$ \frac{\partial R}{\partial v_j} = -2(y - z) \sum_{i=1}^{4} \left( w_i \cdot s'(u_i) \cdot x_{i+j-1} \right) $$
	
	از آنجا که مشتق تابع سیگموئید $s'(x) = s(x)(1-s(x))$ است و $h_i = s(u_i)$، می‌توانیم بنویسیم $s'(u_i) = h_i(1 - h_i)$. فرم نهایی به صورت زیر خواهد بود:
	
	$$ \frac{\partial R}{\partial v_j} = -2(y - z) \sum_{i=1}^{4} w_i h_i(1 - h_i) x_{i+j-1} $$
	
\end{document}