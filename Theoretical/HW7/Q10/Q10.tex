\documentclass[12pt,a4paper]{article}
\usepackage[utf8]{inputenc}
\usepackage{xepersian}
\settextfont{Amiri}
\setlatintextfont{Times New Roman}
\usepackage{amsmath}
\usepackage{amsfonts}
\usepackage{amssymb}
\usepackage{geometry}
\usepackage{enumitem}
\geometry{margin=2.5cm}

\title{تکلیف هفتم درس شناسایی الگو}
\author{وحید ملکی \\ شماره دانشجویی: 40313004}
\date{\today}

\begin{document}
	
	\maketitle
	
	\section*{سؤال 10}
	
	\textbf{صورت سؤال:} الگوریتم پرسپترون با وزن‌های اولیه $w = [1, 1]^T$ و بایاس $b=0$ و نرخ یادگیری $\mu=1$ داده شده است. با توجه به داده‌های جدول، خط جداسازی پس از یک دور (Epoch) مشاهده‌ی تمام نمونه‌ها را بیابید.
	شرط به‌روزرسانی: اگر $y_i(w^T X_i + b) \le 0$ باشد، آپدیت انجام می‌شود.
	
	\subsection*{داده‌های مسأله}
	داده‌های آموزشی به شرح زیر هستند:
	\begin{itemize}
		\item نمونه ۱: $X_1 = \begin{bmatrix} 10 \\ 10 \end{bmatrix}, \quad y_1 = +1$
		\item نمونه ۲: $X_2 = \begin{bmatrix} 0 \\ 0 \end{bmatrix}, \quad y_2 = -1$
		\item نمونه ۳: $X_3 = \begin{bmatrix} 8 \\ 4 \end{bmatrix}, \quad y_3 = +1$
		\item نمونه ۴: $X_4 = \begin{bmatrix} 3 \\ 3 \end{bmatrix}, \quad y_4 = -1$
	\end{itemize}
	
	\subsection*{مراحل اجرا (پیمایش نمونه‌ها)}
	
	\textbf{۱. بررسی نمونه اول $(X_1, y_1)$:}
	مقادیر فعلی: $w = \begin{bmatrix} 1 \\ 1 \end{bmatrix}, b=0$.
	\begin{align*}
		g(X_1) &= w^T X_1 + b = (1 \times 10) + (1 \times 10) + 0 = 20 \\
		\text{بررسی شرط خطا:} &\quad y_1 \times g(X_1) = 1 \times 20 = 20 > 0
	\end{align*}
	چون حاصل مثبت است، دسته‌بندی صحیح بوده و \textbf{نیازی به به‌روزرسانی نیست}.
	
	\vspace{0.5cm}
	
	\textbf{۲. بررسی نمونه دوم $(X_2, y_2)$:}
	مقادیر فعلی: همان مقادیر اولیه.
	\begin{align*}
		g(X_2) &= (1 \times 0) + (1 \times 0) + 0 = 0 \\
		\text{بررسی شرط خطا:} &\quad y_2 \times g(X_2) = -1 \times 0 = 0 \le 0
	\end{align*}
	طبق صورت سؤال، در حالت مساوی با صفر \textbf{به‌روزرسانی انجام می‌شود}:
	\begin{align*}
		w_{\text{new}} &= w_{\text{old}} + \mu y_2 X_2 = \begin{bmatrix} 1 \\ 1 \end{bmatrix} + 1(-1)\begin{bmatrix} 0 \\ 0 \end{bmatrix} = \begin{bmatrix} 1 \\ 1 \end{bmatrix} \\
		b_{\text{new}} &= b_{\text{old}} + \mu y_2 = 0 + 1(-1) = -1
	\end{align*}
	(دقت کنید که چون بردار ورودی صفر بود، بردار وزن تغییری نکرد اما بایاس تغییر یافت).
	
	\vspace{0.5cm}
	
	\textbf{۳. بررسی نمونه سوم $(X_3, y_3)$:}
	مقادیر فعلی: $w = \begin{bmatrix} 1 \\ 1 \end{bmatrix}, b=-1$.
	\begin{align*}
		g(X_3) &= (1 \times 8) + (1 \times 4) + (-1) = 11 \\
		\text{بررسی شرط خطا:} &\quad y_3 \times g(X_3) = 1 \times 11 = 11 > 0
	\end{align*}
	دسته‌بندی صحیح است، \textbf{بدون تغییر}.
	
	\vspace{0.5cm}
	
	\textbf{۴. بررسی نمونه چهارم $(X_4, y_4)$:}
	مقادیر فعلی: $w = \begin{bmatrix} 1 \\ 1 \end{bmatrix}, b=-1$.
	\begin{align*}
		g(X_4) &= (1 \times 3) + (1 \times 3) + (-1) = 5 \\
		\text{بررسی شرط خطا:} &\quad y_4 \times g(X_4) = -1 \times 5 = -5 \le 0
	\end{align*}
	خطا رخ داده است. \textbf{به‌روزرسانی انجام می‌شود}:
	\begin{align*}
		w_{\text{new}} &= w_{\text{old}} + \mu y_4 X_4 = \begin{bmatrix} 1 \\ 1 \end{bmatrix} + 1(-1)\begin{bmatrix} 3 \\ 3 \end{bmatrix} = \begin{bmatrix} 1-3 \\ 1-3 \end{bmatrix} = \begin{bmatrix} -2 \\ -2 \end{bmatrix} \\
		b_{\text{new}} &= b_{\text{old}} + \mu y_4 = -1 + 1(-1) = -2
	\end{align*}
	
	\subsection*{پاسخ نهایی}
	پس از مشاهده‌ی تمام نمونه‌ها، پارامترهای نهایی شبکه عبارتند از:
	$$ w = \begin{bmatrix} -2 \\ -2 \end{bmatrix}, \quad b = -2 $$
	معادله‌ی خط جداکننده ($g(x)=0$) به صورت زیر خواهد بود:
	$$ -2x_1 - 2x_2 - 2 = 0 $$
	که با ساده‌سازی (تقسیم بر $-2$) به معادله‌ی زیر می‌رسیم:
	\begin{equation*}
		\boxed{x_1 + x_2 + 1 = 0}
	\end{equation*}
	
\end{document}